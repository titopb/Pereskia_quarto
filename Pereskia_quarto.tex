% Options for packages loaded elsewhere
\PassOptionsToPackage{unicode}{hyperref}
\PassOptionsToPackage{hyphens}{url}
\PassOptionsToPackage{dvipsnames,svgnames,x11names}{xcolor}
%
\documentclass[
  12pt,
  letterpaper,
  DIV=11,
  numbers=noendperiod]{scrartcl}

\usepackage{amsmath,amssymb}
\usepackage{iftex}
\ifPDFTeX
  \usepackage[T1]{fontenc}
  \usepackage[utf8]{inputenc}
  \usepackage{textcomp} % provide euro and other symbols
\else % if luatex or xetex
  \usepackage{unicode-math}
  \defaultfontfeatures{Scale=MatchLowercase}
  \defaultfontfeatures[\rmfamily]{Ligatures=TeX,Scale=1}
\fi
\usepackage{lmodern}
\ifPDFTeX\else  
    % xetex/luatex font selection
\fi
% Use upquote if available, for straight quotes in verbatim environments
\IfFileExists{upquote.sty}{\usepackage{upquote}}{}
\IfFileExists{microtype.sty}{% use microtype if available
  \usepackage[]{microtype}
  \UseMicrotypeSet[protrusion]{basicmath} % disable protrusion for tt fonts
}{}
\makeatletter
\@ifundefined{KOMAClassName}{% if non-KOMA class
  \IfFileExists{parskip.sty}{%
    \usepackage{parskip}
  }{% else
    \setlength{\parindent}{0pt}
    \setlength{\parskip}{6pt plus 2pt minus 1pt}}
}{% if KOMA class
  \KOMAoptions{parskip=half}}
\makeatother
\usepackage{xcolor}
\usepackage[margin=1.0in]{geometry}
\setlength{\emergencystretch}{3em} % prevent overfull lines
\setcounter{secnumdepth}{-\maxdimen} % remove section numbering
% Make \paragraph and \subparagraph free-standing
\ifx\paragraph\undefined\else
  \let\oldparagraph\paragraph
  \renewcommand{\paragraph}[1]{\oldparagraph{#1}\mbox{}}
\fi
\ifx\subparagraph\undefined\else
  \let\oldsubparagraph\subparagraph
  \renewcommand{\subparagraph}[1]{\oldsubparagraph{#1}\mbox{}}
\fi


\providecommand{\tightlist}{%
  \setlength{\itemsep}{0pt}\setlength{\parskip}{0pt}}\usepackage{longtable,booktabs,array}
\usepackage{calc} % for calculating minipage widths
% Correct order of tables after \paragraph or \subparagraph
\usepackage{etoolbox}
\makeatletter
\patchcmd\longtable{\par}{\if@noskipsec\mbox{}\fi\par}{}{}
\makeatother
% Allow footnotes in longtable head/foot
\IfFileExists{footnotehyper.sty}{\usepackage{footnotehyper}}{\usepackage{footnote}}
\makesavenoteenv{longtable}
\usepackage{graphicx}
\makeatletter
\def\maxwidth{\ifdim\Gin@nat@width>\linewidth\linewidth\else\Gin@nat@width\fi}
\def\maxheight{\ifdim\Gin@nat@height>\textheight\textheight\else\Gin@nat@height\fi}
\makeatother
% Scale images if necessary, so that they will not overflow the page
% margins by default, and it is still possible to overwrite the defaults
% using explicit options in \includegraphics[width, height, ...]{}
\setkeys{Gin}{width=\maxwidth,height=\maxheight,keepaspectratio}
% Set default figure placement to htbp
\makeatletter
\def\fps@figure{htbp}
\makeatother
\newlength{\cslhangindent}
\setlength{\cslhangindent}{1.5em}
\newlength{\csllabelwidth}
\setlength{\csllabelwidth}{3em}
\newlength{\cslentryspacingunit} % times entry-spacing
\setlength{\cslentryspacingunit}{\parskip}
\newenvironment{CSLReferences}[2] % #1 hanging-ident, #2 entry spacing
 {% don't indent paragraphs
  \setlength{\parindent}{0pt}
  % turn on hanging indent if param 1 is 1
  \ifodd #1
  \let\oldpar\par
  \def\par{\hangindent=\cslhangindent\oldpar}
  \fi
  % set entry spacing
  \setlength{\parskip}{#2\cslentryspacingunit}
 }%
 {}
\usepackage{calc}
\newcommand{\CSLBlock}[1]{#1\hfill\break}
\newcommand{\CSLLeftMargin}[1]{\parbox[t]{\csllabelwidth}{#1}}
\newcommand{\CSLRightInline}[1]{\parbox[t]{\linewidth - \csllabelwidth}{#1}\break}
\newcommand{\CSLIndent}[1]{\hspace{\cslhangindent}#1}

\KOMAoption{captions}{tableheading}
\usepackage{helvet}
\renewcommand*\familydefault{\sfdefault}
\usepackage{setspace}
\usepackage[left]{lineno}
\linenumbers
\makeatletter
\makeatother
\makeatletter
\makeatother
\makeatletter
\@ifpackageloaded{caption}{}{\usepackage{caption}}
\AtBeginDocument{%
\ifdefined\contentsname
  \renewcommand*\contentsname{Table of contents}
\else
  \newcommand\contentsname{Table of contents}
\fi
\ifdefined\listfigurename
  \renewcommand*\listfigurename{List of Figures}
\else
  \newcommand\listfigurename{List of Figures}
\fi
\ifdefined\listtablename
  \renewcommand*\listtablename{List of Tables}
\else
  \newcommand\listtablename{List of Tables}
\fi
\ifdefined\figurename
  \renewcommand*\figurename{Figure}
\else
  \newcommand\figurename{Figure}
\fi
\ifdefined\tablename
  \renewcommand*\tablename{Table}
\else
  \newcommand\tablename{Table}
\fi
}
\@ifpackageloaded{float}{}{\usepackage{float}}
\floatstyle{ruled}
\@ifundefined{c@chapter}{\newfloat{codelisting}{h}{lop}}{\newfloat{codelisting}{h}{lop}[chapter]}
\floatname{codelisting}{Listing}
\newcommand*\listoflistings{\listof{codelisting}{List of Listings}}
\makeatother
\makeatletter
\@ifpackageloaded{caption}{}{\usepackage{caption}}
\@ifpackageloaded{subcaption}{}{\usepackage{subcaption}}
\makeatother
\makeatletter
\@ifpackageloaded{tcolorbox}{}{\usepackage[skins,breakable]{tcolorbox}}
\makeatother
\makeatletter
\@ifundefined{shadecolor}{\definecolor{shadecolor}{rgb}{.97, .97, .97}}
\makeatother
\makeatletter
\makeatother
\makeatletter
\makeatother
\ifLuaTeX
  \usepackage{selnolig}  % disable illegal ligatures
\fi
\IfFileExists{bookmark.sty}{\usepackage{bookmark}}{\usepackage{hyperref}}
\IfFileExists{xurl.sty}{\usepackage{xurl}}{} % add URL line breaks if available
\urlstyle{same} % disable monospaced font for URLs
\hypersetup{
  colorlinks=true,
  linkcolor={blue},
  filecolor={Maroon},
  citecolor={Blue},
  urlcolor={Blue},
  pdfcreator={LaTeX via pandoc}}

\author{}
\date{}

\begin{document}
\ifdefined\Shaded\renewenvironment{Shaded}{\begin{tcolorbox}[frame hidden, sharp corners, interior hidden, breakable, boxrule=0pt, borderline west={3pt}{0pt}{shadecolor}, enhanced]}{\end{tcolorbox}}\fi

\hypertarget{surviving-climate-change-understanding-the-facultative-cam-photosynthesis-in-pereskia-aculeata-plants}{%
\section{\texorpdfstring{Surviving climate change: Understanding the
facultative CAM photosynthesis in \emph{Pereskia aculeata}
plants}{Surviving climate change: Understanding the facultative CAM photosynthesis in Pereskia aculeata plants}}\label{surviving-climate-change-understanding-the-facultative-cam-photosynthesis-in-pereskia-aculeata-plants}}

\textbf{Runing title:} Facultative CAM photosynthesis in \emph{Pereskia
aculeata}

Luanna Costa Cenciareli\({^1}\); Moni Soares Justi\({^1}\); Sérgio Luiz
Ferreira da Silva\({^2}\); João Paulo Alves de Barros\({^2}\); Milton C.
Lima Neto\({^1}{^\star}\)

\({^1}\)Institute of Biosciences, Coastal Campus,State University of São
Paulo - UNESP. São Vicente, SP, Brazil. \({^2}\)Academic Unity of Serra
Talhada, Federal Rural University of Pernambuco - UFRPE, Serra Talhada,
PE, Brazil.

\({^\star}\)To whom correspondence should be addressed:
\href{mailto:milton.lima-neto@unesp.br}{milton.lima-neto@unesp.br}
Institute of Biosciences, Coastal Campus. State University of São Paulo,
P.O Box: 73601/SP ZIPCODE: 11380-972. São Vicente, SP, Brazil.

\vspace{2mm}

\hypertarget{abstract}{%
\subsection{Abstract}\label{abstract}}

Keywords:

\newpage

\hypertarget{introduction}{%
\subsection{Introduction}\label{introduction}}

Climate change has a significant impact on global plant biodiversity and
food security, with increasing temperatures, droughts, and extreme
weather events posing challenges to plant growth and crop production
(Fenollosa and Munné-Bosch, 2019). These changes threaten to reduce crop
yields and disturb ecosystems, resulting in an increased risk of food
insecurity and harm to communities worldwide. Therefore, it is essential
to better comprehend the impacts of global warming on plant biodiversity
and food security and to develop effective strategies for mitigating
these effects, ensuring agriculture and natural ecosystems have a
sustainable future (Rosenzweig et al., 2014). Changes in development
patterns, including flowering and fruiting periods, as well as increased
susceptibility to parasites and diseases, are among the diverse effects
of climate change on plants.

Climate change is leading to a harsher and drier planet, and as a
result, the demand for water in agriculture is expected to double by
2050, while the availability of fresh water is projected to decrease by
50\% ((Gupta et al., 2020). Drought alone causes more annual losses in
crop yield than all pathogens combined (Gupta et al., 2020). Water is
essential for plant survival, and water deficits reduce plant growth.
Drought is a common effect of climate change that negatively impacts
plant productivity and growth by disrupting the energy balance between
the capacity to absorb and use light energy during photosynthesis. It
damages the physiological metabolism and photosynthesis of plants,
ultimately reducing crop production. C3 plants typically close stomata
under drought conditions to avoid water loss, decreasing \(CO_2\)
assimilation (Lawson and Vialet-Chabrand, 2019). Since the
Calvin-Benson-Bassham cycle is one of the primary electron sinks of
photosynthetic electron transport, stomatal limitations can upset the
delicate balance between the production and consumption of reducing
equivalents such as \(Fd_{red}\), ATP, and NADPH, producing excess
energy in chloroplasts (Lima Neto et al., 2017). This imbalance disrupts
the redox equilibrium in the chloroplast, thereby altering the ROS
processing and signaling systems (Foyer, 2018).

Plants have evolved strategies to prevent water loss, including but not
limited to closing stomata and developing deeper roots (Chaves et al.,
2003). However, there is still much work to be done to produce crops
that can thrive in a changing climate. Plant breeding and biotechnology
are key areas where progress can be made to develop crops that are more
resilient and adaptable to changing environmental conditions, such as
heat and drought stress (Gupta et al., 2020). One potential adaptation
strategy for plants to cope with the challenges of climate change is
crassulacean acid metabolism (CAM), a unique type of photosynthesis that
improves water-use efficiency (WUE) and drought tolerance
(\textbf{luttge\_ecophysiology\_2004?}). CAM is a unique photosynthetic
pathway that has evolved in plants growing in arid and semi-arid
environments. CAM plants have adapted to water-limited conditions by
fixing \(CO_2\) at night when the stomata are open, storing it as
organic acids, and releasing it during the day when the stomata are
closed (\textbf{luttge\_ecophysiology\_2004?}). Despite the significance
of CAM metabolism, its regulation, molecular basis, and evolution, as
well as the mechanisms underlying mesophyll conductance and the
regulation of photosynthetic electron transport in these plants, are not
completely understood (Cousins et al., 2020). Moreover, studies have
also been conducted on the photosynthetic flexibility of facultative CAM
plants, which can switch between CAM and C3 or C4 photosynthesis
depending on environmental conditions (Winter and Holtum, 2014). Thus,
understanding the physiological mechanisms underlying CAM photosynthesis
is essential for developing strategies to improve crop productivity in
water-limited environments and for predicting the responses of natural
ecosystems to climate change.

\emph{Pereskia aculeata} Mill., also known as ora-pro-nobis or Barbados
gooseberry, is a highly nutritious plant species from the Cactaceae
family, subfamily Pereskioideae. It is an edible plant commonly found in
the Brazilian Atlantic Forest and is known for its high mineral and
protein contents (Queiroz et al., 2015). \emph{P. aculeata} is a hardy
and perennial plant that can grow well in various soil types. However,
it has become a problematic invasive alien plant in other regions due to
its plasticity and stress tolerance (Venter et al., 2022).

However, the provided research results do not provide any information on
the facultative Crassulacean Acid Metabolism (CAM) photosynthesis in P.
aculeata plants. Therefore, further research is needed to understand the
CAM photosynthesis in P. aculeata plants and its potential role in
helping the species to survive in changing environmental conditions
caused by climate change.

(Luttge, 2004)

\hypertarget{material-and-methods}{%
\subsection{Material and methods}\label{material-and-methods}}

\hypertarget{experimental-design-and-plant-growth}{%
\subsubsection{Experimental design and plant
growth}\label{experimental-design-and-plant-growth}}

\hypertarget{other-topics-of-mm}{%
\subsubsection{other topics of M\&M}\label{other-topics-of-mm}}

\hypertarget{results}{%
\subsection{Results}\label{results}}

\hypertarget{discussion}{%
\subsection{Discussion}\label{discussion}}

\begin{itemize}
\tightlist
\item
  Briefly synthesize results
\end{itemize}

\hypertarget{conclusion}{%
\subsection{Conclusion}\label{conclusion}}

\hypertarget{ackonwledgments}{%
\subsection{Ackonwledgments}\label{ackonwledgments}}

\clearpage

\hypertarget{references}{%
\subsection{References}\label{references}}

\hypertarget{refs}{}
\begin{CSLReferences}{1}{0}
\leavevmode\vadjust pre{\hypertarget{ref-chaves_understanding_2003}{}}%
Chaves, M.M., Maroco, J.P., Pereira, J.S., 2003. Understanding plant
responses to drought \textemdash{} from genes to the whole plant.
Functional Plant Biology 30, 239. \url{https://doi.org/10.1071/FP02076}

\leavevmode\vadjust pre{\hypertarget{ref-cousins2020}{}}%
Cousins, A.B., Mullendore, D.L., Sonawane, B.V., 2020. Recent
developments in mesophyll conductance in C3, C4, and crassulacean acid
metabolism plants. The Plant Journal 101, 816--830.
\url{https://doi.org/10.1111/tpj.14664}

\leavevmode\vadjust pre{\hypertarget{ref-fenollosa2019}{}}%
Fenollosa, E., Munné-Bosch, S., 2019. Physiological plasticity of plants
facing climate change. Annual plant reviews online 2, 1--29.
\url{https://doi.org/10.1002/9781119312994.APR0686}

\leavevmode\vadjust pre{\hypertarget{ref-foyer_reactive_2018}{}}%
Foyer, C.H., 2018. Reactive oxygen species, oxidative signaling and the
regulation of photosynthesis. Environmental and Experimental Botany 154,
134--142. \url{https://doi.org/10.1016/j.envexpbot.2018.05.003}

\leavevmode\vadjust pre{\hypertarget{ref-gupta2020}{}}%
Gupta, A., Rico-Medina, A., Caño-Delgado, A.I., 2020. The physiology of
plant responses to drought. Science 368, 266--269.
\url{https://doi.org/10.1126/science.aaz7614}

\leavevmode\vadjust pre{\hypertarget{ref-lawson2019}{}}%
Lawson, T., Vialet-Chabrand, S., 2019. Speedy stomata, photosynthesis
and plant water use efficiency. New Phytologist 221, 93--98.
\url{https://doi.org/10.1111/nph.15330}

\leavevmode\vadjust pre{\hypertarget{ref-lima_neto_regulation_2017}{}}%
Lima Neto, M.C., Silveira, J.A.G., Cerqueira, J.V.A., Cunha, J.R., 2017.
Regulation of the photosynthetic electron transport and specific
photoprotective mechanisms in {Ricinus} communis under drought and
recovery. Acta Physiologiae Plantarum 39, 183.
\url{https://doi.org/10.1007/s11738-017-2483-9}

\leavevmode\vadjust pre{\hypertarget{ref-luttge2004}{}}%
Luttge, U., 2004. Ecophysiology of Crassulacean Acid Metabolism (CAM).
Annals of Botany 93, 629--652. \url{https://doi.org/10.1093/aob/mch087}

\leavevmode\vadjust pre{\hypertarget{ref-queiroz_growing_2015}{}}%
Queiroz, C.R.A. dos A., de Andrade, R.R., de Morais, S.A.L., Pavani,
L.C., 2015. Growing {Pereskia} aculeata under intermittent irrigation
according to levels of matric potential reduction. Tropical agricultural
research 45, 1--8. \url{https://doi.org/10.1590/1983-40632015v4527210}

\leavevmode\vadjust pre{\hypertarget{ref-rosenzweig2014}{}}%
Rosenzweig, C., Elliott, J., Deryng, D., Ruane, A.C., Müller, C.,
Arneth, A., Boote, K.J., Folberth, C., Glotter, M., Khabarov, N.,
Neumann, K., Piontek, F., Pugh, T.A.M., Schmid, E., Stehfest, E., Yang,
H., Jones, J.W., 2014. Assessing agricultural risks of climate change in
the 21st century in a global gridded crop model intercomparison.
Proceedings of the National Academy of Sciences 111, 3268--3273.
\url{https://doi.org/10.1073/pnas.1222463110}

\leavevmode\vadjust pre{\hypertarget{ref-venter_interactive_2022}{}}%
Venter, N., Cowie, B.W., Paterson, I.D., Witkowski, E.T.F., Byrne, M.J.,
2022. The interactive effects of {CO2} and water on the growth and
physiology of the invasive alien vine {Pereskia} aculeata ({Cactaceae}):
{Implications} for its future invasion and management. Environmental and
Experimental Botany 194, 104737.
\url{https://doi.org/10.1016/j.envexpbot.2021.104737}

\leavevmode\vadjust pre{\hypertarget{ref-winter_facultative_2014}{}}%
Winter, K., Holtum, J.A.M., 2014. Facultative crassulacean acid
metabolism ({CAM}) plants: {Powerful} tools for unravelling the
functional elements of {CAM} photosynthesis. Journal of Experimental
Botany 65, 3425--3441. \url{https://doi.org/10.1093/jxb/eru063}

\end{CSLReferences}

\newpage

\hypertarget{tables}{%
\subsection{Tables}\label{tables}}

\hypertarget{figures}{%
\subsection{Figures}\label{figures}}

\textbf{Figure 1.} The caption of figure 1

\newpage

\textbf{Figure 2.} The caption of figure 2



\end{document}
