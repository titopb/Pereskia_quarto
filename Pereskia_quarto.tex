% Options for packages loaded elsewhere
\PassOptionsToPackage{unicode}{hyperref}
\PassOptionsToPackage{hyphens}{url}
\PassOptionsToPackage{dvipsnames,svgnames,x11names}{xcolor}
%
\documentclass[
  12pt,
  letterpaper,
  DIV=11,
  numbers=noendperiod]{scrartcl}

\usepackage{amsmath,amssymb}
\usepackage{iftex}
\ifPDFTeX
  \usepackage[T1]{fontenc}
  \usepackage[utf8]{inputenc}
  \usepackage{textcomp} % provide euro and other symbols
\else % if luatex or xetex
  \usepackage{unicode-math}
  \defaultfontfeatures{Scale=MatchLowercase}
  \defaultfontfeatures[\rmfamily]{Ligatures=TeX,Scale=1}
\fi
\usepackage{lmodern}
\ifPDFTeX\else  
    % xetex/luatex font selection
\fi
% Use upquote if available, for straight quotes in verbatim environments
\IfFileExists{upquote.sty}{\usepackage{upquote}}{}
\IfFileExists{microtype.sty}{% use microtype if available
  \usepackage[]{microtype}
  \UseMicrotypeSet[protrusion]{basicmath} % disable protrusion for tt fonts
}{}
\makeatletter
\@ifundefined{KOMAClassName}{% if non-KOMA class
  \IfFileExists{parskip.sty}{%
    \usepackage{parskip}
  }{% else
    \setlength{\parindent}{0pt}
    \setlength{\parskip}{6pt plus 2pt minus 1pt}}
}{% if KOMA class
  \KOMAoptions{parskip=half}}
\makeatother
\usepackage{xcolor}
\usepackage[margin=1.0in]{geometry}
\setlength{\emergencystretch}{3em} % prevent overfull lines
\setcounter{secnumdepth}{-\maxdimen} % remove section numbering
% Make \paragraph and \subparagraph free-standing
\ifx\paragraph\undefined\else
  \let\oldparagraph\paragraph
  \renewcommand{\paragraph}[1]{\oldparagraph{#1}\mbox{}}
\fi
\ifx\subparagraph\undefined\else
  \let\oldsubparagraph\subparagraph
  \renewcommand{\subparagraph}[1]{\oldsubparagraph{#1}\mbox{}}
\fi


\providecommand{\tightlist}{%
  \setlength{\itemsep}{0pt}\setlength{\parskip}{0pt}}\usepackage{longtable,booktabs,array}
\usepackage{calc} % for calculating minipage widths
% Correct order of tables after \paragraph or \subparagraph
\usepackage{etoolbox}
\makeatletter
\patchcmd\longtable{\par}{\if@noskipsec\mbox{}\fi\par}{}{}
\makeatother
% Allow footnotes in longtable head/foot
\IfFileExists{footnotehyper.sty}{\usepackage{footnotehyper}}{\usepackage{footnote}}
\makesavenoteenv{longtable}
\usepackage{graphicx}
\makeatletter
\def\maxwidth{\ifdim\Gin@nat@width>\linewidth\linewidth\else\Gin@nat@width\fi}
\def\maxheight{\ifdim\Gin@nat@height>\textheight\textheight\else\Gin@nat@height\fi}
\makeatother
% Scale images if necessary, so that they will not overflow the page
% margins by default, and it is still possible to overwrite the defaults
% using explicit options in \includegraphics[width, height, ...]{}
\setkeys{Gin}{width=\maxwidth,height=\maxheight,keepaspectratio}
% Set default figure placement to htbp
\makeatletter
\def\fps@figure{htbp}
\makeatother
\newlength{\cslhangindent}
\setlength{\cslhangindent}{1.5em}
\newlength{\csllabelwidth}
\setlength{\csllabelwidth}{3em}
\newlength{\cslentryspacingunit} % times entry-spacing
\setlength{\cslentryspacingunit}{\parskip}
\newenvironment{CSLReferences}[2] % #1 hanging-ident, #2 entry spacing
 {% don't indent paragraphs
  \setlength{\parindent}{0pt}
  % turn on hanging indent if param 1 is 1
  \ifodd #1
  \let\oldpar\par
  \def\par{\hangindent=\cslhangindent\oldpar}
  \fi
  % set entry spacing
  \setlength{\parskip}{#2\cslentryspacingunit}
 }%
 {}
\usepackage{calc}
\newcommand{\CSLBlock}[1]{#1\hfill\break}
\newcommand{\CSLLeftMargin}[1]{\parbox[t]{\csllabelwidth}{#1}}
\newcommand{\CSLRightInline}[1]{\parbox[t]{\linewidth - \csllabelwidth}{#1}\break}
\newcommand{\CSLIndent}[1]{\hspace{\cslhangindent}#1}

\KOMAoption{captions}{tableheading}
\usepackage{soul}
\renewcommand*\familydefault{\sfdefault}
\usepackage{setspace}
\usepackage[left]{lineno}
\linenumbers
\makeatletter
\makeatother
\makeatletter
\makeatother
\makeatletter
\@ifpackageloaded{caption}{}{\usepackage{caption}}
\AtBeginDocument{%
\ifdefined\contentsname
  \renewcommand*\contentsname{Table of contents}
\else
  \newcommand\contentsname{Table of contents}
\fi
\ifdefined\listfigurename
  \renewcommand*\listfigurename{List of Figures}
\else
  \newcommand\listfigurename{List of Figures}
\fi
\ifdefined\listtablename
  \renewcommand*\listtablename{List of Tables}
\else
  \newcommand\listtablename{List of Tables}
\fi
\ifdefined\figurename
  \renewcommand*\figurename{Figure}
\else
  \newcommand\figurename{Figure}
\fi
\ifdefined\tablename
  \renewcommand*\tablename{Table}
\else
  \newcommand\tablename{Table}
\fi
}
\@ifpackageloaded{float}{}{\usepackage{float}}
\floatstyle{ruled}
\@ifundefined{c@chapter}{\newfloat{codelisting}{h}{lop}}{\newfloat{codelisting}{h}{lop}[chapter]}
\floatname{codelisting}{Listing}
\newcommand*\listoflistings{\listof{codelisting}{List of Listings}}
\makeatother
\makeatletter
\@ifpackageloaded{caption}{}{\usepackage{caption}}
\@ifpackageloaded{subcaption}{}{\usepackage{subcaption}}
\makeatother
\makeatletter
\@ifpackageloaded{tcolorbox}{}{\usepackage[skins,breakable]{tcolorbox}}
\makeatother
\makeatletter
\@ifundefined{shadecolor}{\definecolor{shadecolor}{rgb}{.97, .97, .97}}
\makeatother
\makeatletter
\makeatother
\makeatletter
\makeatother
\ifLuaTeX
\usepackage[bidi=basic]{babel}
\else
\usepackage[bidi=default]{babel}
\fi
\babelprovide[main,import]{english}
% get rid of language-specific shorthands (see #6817):
\let\LanguageShortHands\languageshorthands
\def\languageshorthands#1{}
\ifLuaTeX
  \usepackage{selnolig}  % disable illegal ligatures
\fi
\IfFileExists{bookmark.sty}{\usepackage{bookmark}}{\usepackage{hyperref}}
\IfFileExists{xurl.sty}{\usepackage{xurl}}{} % add URL line breaks if available
\urlstyle{same} % disable monospaced font for URLs
\hypersetup{
  pdflang={en},
  colorlinks=true,
  linkcolor={blue},
  filecolor={Maroon},
  citecolor={Blue},
  urlcolor={Blue},
  pdfcreator={LaTeX via pandoc}}

\author{}
\date{}

\begin{document}
\ifdefined\Shaded\renewenvironment{Shaded}{\begin{tcolorbox}[enhanced, frame hidden, breakable, borderline west={3pt}{0pt}{shadecolor}, sharp corners, boxrule=0pt, interior hidden]}{\end{tcolorbox}}\fi

\hypertarget{surviving-climate-change-understanding-the-facultative-cam-photosynthesis-in-pereskia-aculeata-plants}{%
\section{\texorpdfstring{Surviving climate change: Understanding the
facultative CAM photosynthesis in \emph{Pereskia aculeata}
plants}{Surviving climate change: Understanding the facultative CAM photosynthesis in Pereskia aculeata plants}}\label{surviving-climate-change-understanding-the-facultative-cam-photosynthesis-in-pereskia-aculeata-plants}}

\textbf{Runing title:} Facultative CAM photosynthesis in \emph{Pereskia
aculeata}

Luanna Costa Cenciareli\({^1}\); Moni Soares Justi\({^1}\); Sérgio Luiz
Ferreira da Silva\({^2}\); João Paulo Alves de Barros\({^2}\); Milton C.
Lima Neto\({^1}{^\star}\)

\({^1}\)Institute of Biosciences, Coastal Campus,State University of São
Paulo - UNESP. São Vicente, SP, Brazil. \({^2}\)Academic Unity of Serra
Talhada, Federal Rural University of Pernambuco - UFRPE, Serra Talhada,
PE, Brazil.

\({^\star}\)To whom correspondence should be addressed:
\href{mailto:milton.lima-neto@unesp.br}{milton.lima-neto@unesp.br}
Institute of Biosciences, Coastal Campus. State University of São Paulo,
P.O Box: 73601/SP ZIPCODE: 11380-972. São Vicente, SP, Brazil.

\vspace{2mm}

\hypertarget{abstract}{%
\subsection{Abstract}\label{abstract}}

Keywords:

Abbreviations: \(\psi_w\) - water potential, PPFD - photosynthetic
photon flux density, PVPP - polivinilpolipirrolidone, rcf - relative
centrifuge force, TBARS - thiobarbituric acid reactive substances;

\newpage

\hypertarget{introduction}{%
\subsection{Introduction}\label{introduction}}

Climate change has a significant impact on global plant biodiversity and
food security, with increasing temperatures, droughts, and extreme
weather events posing challenges to plant growth and crop production
(Fenollosa and Munné-Bosch, 2019). These changes threaten to reduce crop
yields and disturb ecosystems, resulting in an increased risk of food
insecurity and harm to communities worldwide. Therefore, it is essential
to better comprehend the impacts of global warming on plant biodiversity
and food security and to develop effective strategies for mitigating
these effects, ensuring agriculture and natural ecosystems have a
sustainable future (Rosenzweig et al., 2014). Changes in development
patterns, including flowering and fruiting periods, as well as increased
susceptibility to parasites and diseases, are among the diverse effects
of climate change on plants.

Climate change is leading to a harsher and drier planet, and as a
result, the demand for water in agriculture is expected to double by
2050, while the availability of fresh water is projected to decrease by
50\% (Gupta et al., 2020). Drought alone causes more annual losses in
crop yield than all pathogens combined (Gupta et al., 2020). Water is
essential for plant survival, and water deficits reduce plant growth.
Drought is a common effect of climate change that negatively impacts
plant productivity and growth by disrupting the energy balance between
the capacity to absorb and use light energy during photosynthesis. It
damages the physiological metabolism and photosynthesis of plants,
ultimately reducing crop production. C3 plants typically close stomata
under drought conditions to avoid water loss, decreasing \(CO_2\)
assimilation (Lawson and Vialet-Chabrand, 2019). Since the
Calvin-Benson-Bassham cycle is one of the primary electron sinks of
photosynthetic electron transport, stomatal limitations can upset the
delicate balance between the production and consumption of reducing
equivalents such as \(Fd_{red}\), ATP, and NADPH, producing excess
energy in chloroplasts (Lima Neto et al., 2017). This imbalance disrupts
the redox equilibrium in the chloroplast, thereby altering the reactive
oxygen species (ROS) processing and signaling systems (Foyer, 2018).

Plants have strategies to prevent water loss, including but not limited
to closing stomata and developing deeper roots (Chaves et al., 2003).
However, there is still much work to be done to produce crops that can
thrive in a changing climate. Plant breeding and biotechnology are key
areas where progress can be made to develop crops that are more
resilient and adaptable to changing environmental conditions such as
heat and drought stress (Gupta et al., 2020). One potential adaptation
strategy for plants to cope with the challenges of climate change is
crassulacean acid metabolism (CAM), a unique type of photosynthesis that
improves water-use efficiency (WUE) and drought tolerance (Luttge,
2004). CAM is a distinctive photosynthetic pathway that has evolved in
plants growing in arid and semi-arid environments.

CAM plants have adapted to water-limited conditions by fixing \(CO_2\)
at night when the stomata are open, storing it as organic acids, and
releasing it during the day when the stomata are closed (Luttge, 2004).
This process, known as obligate CAM, is characterized by the strict
separation of day and night metabolic activities. In Phase I, which
occurs during the dark period, \(CO_2\) is fixed by phosphoenolpyruvate
carboxylase (PEPC) to form malate, which is then accumulated in the
vacuole. In Phase II there is a transition from nocturnal to daytime
conditions. The stomata remain open for a short time during the very
early light period, then close as malic acid is decarboxylated to
release \(CO_2\) for photosynthesis. During Phase III, the stomata close
completely, and the stored malate is transported out of the vacuole. The
malate is then decarboxylated to release \(CO_2\), which is assimilated
via Calvin-Benson-Bassham (CBB) cycle reactions. In Phase IV, the
transition from daytime to nocturnal conditions occurs. When the malate
supply is exhausted in the late afternoon, the stomata reopen for
\(CO_2\) uptake with direct assimilation (Luttge, 2004)

The expression of CAM phases is plastic, affecting CAM terminology and
definitions (Luttge, 2004). First, CAM-idling, often seen under extreme
drought conditions, involves nocturnal acid accumulation without daytime
decarboxylation, essentially a ``survival mode'', where stomata remain
closed day and night, and the organic cycle is fed by internal recycling
of nocturnaly refixed respiratory \(CO_2\). CAM-cycling, where stomata
remain closed during the dark period with some minor synthesis of
organic acid fed by respiratory \(CO_2\). Stomata are open during the
light period, with uptake of atmospheric \(CO_2\) and direct
assimilation of CBB reactions in addition to the assimilation of
remobilized \(CO_2\) from noctunarly stored organic acid (Luttge, 2004).
Finally, facultative CAM allows a dynamic switch between \(C_3\) and CAM
modes depending on environmental conditions Winter and Holtum (2014).

The different species may present various forms of CAM and may also be
expressed temporarily in a given species, constituting different CAM
physiotypes (Winter and Holtum, 2014). Despite the significance of CAM
metabolism, its regulation, molecular basis, and evolution, as well as
the mechanisms underlying mesophyll conductance and the regulation of
photosynthetic electron transport in these plants, are not completely
understood (Cousins et al., 2020). Moreover, a comprehensive and
coherent definition of CAM is not straightforward. Thus, understanding
the physiological mechanisms underlying CAM photosynthesis is essential
for developing strategies to improve crop productivity in water-limited
environments and for predicting the responses of natural ecosystems to
climate change.

\emph{Pereskia aculeata} Mill., also known as ora-pro-nobis or Barbados
gooseberry, is a highly nutritious leafy plant species from the
Cactaceae family, subfamily Pereskioideae. It is an edible plant
commonly found in the Brazilian Atlantic Forest and is known for its
high mineral and protein contents (Queiroz et al., 2015). \emph{P.
aculeata} is a hardy and perennial plant that can grow well in various
soil types. In contrast to other Pereskia species, which are shrubs or
trees, \emph{P. aculeata} is classified as a liana (woody vine) owing to
its semi-scandent growth form (Venter et al., 2022). It is becoming a
problematic invasive alien plant in South Africa, occurring in high
densities, smothering native trees and shrubs, competing for resources,
and ultimately causing the demise of the affected vegetation (Venter et
al., 2022).

While most Cactaceae species use the obligate CAM photosynthetic pathway
(Luttge, 2004), Nobel and Hartsock (1987){]}, \emph{P. aculeata} is
apparently capable of facultative CAM-cycling (Martin and Wallace,
2000), and drought-induced CAM-idling (Edwards and Diaz, 2006). However,
this interpretation is controversial in the literature and more research
is required. Despite Nobel and Hartsock (1987) showing that drought
stress served only to reduce daytime \(CO_2\) uptake rates, with no
nighttime uptake induced in \emph{P. aculeata}, the authors did not
measure changes in leaf acidity. In the research of Rayder and Ting
(1981), they reported substantial diel changes in acidity of the leaves
of well-watered and drought-stressed \emph{P. aculeata} plants.
Considering the lack of information on the regulation and induction of
facultative CAM photosynthesis, the aim of our study was to evaluate the
influence of drought on the dynamics of CAM metabolism in \emph{P.
aculeata} plants. In particular, we were interested in how \emph{P.
aculeata} adjust the components associated with the light-dark patterns
of light energy use, the regulation of photosynthetic quantum yields,
and photoprotective mechanisms.

\hypertarget{material-and-methods}{%
\subsection{Material and methods}\label{material-and-methods}}

\hypertarget{plant-material-and-growth-conditions}{%
\subsubsection{Plant material and growth
conditions}\label{plant-material-and-growth-conditions}}

\emph{Pereskia aculeata} plants were propagated from cuttings in 15 L
plastic pots, using a 1:1 (v/v) mixture of washed sand and commercial
substrate. They were maintained under greenhouse conditions at the
Institute of Biosciences, São Paulo State University, São Vicente,
Brazil. The plants were watered daily to pot capacity and supplemented
weekly with a half-strength Hoagland and Arnon nutrient solution
(Hoagland and Arnon, 1950). After three months, when the plants were
well-established and growing vigorously, they were divided into two
groups. The first group continued to be watered daily to pot capacity
(control), while the second group was watered with only 30\% of the
volume used to achieve pot capacity (drought). Water-stressed plants
showed visible wilting and chlorosis at the time of experimentation. At
this point, plants were harvested for future analysis. One group of
plants was kept under control conditions, while another group that had
been exposed to 90 days of drought was re-watered to pot capacity for 7
days (recovery). After the recovery period, plants were harvested for
future analysis. Five plants were used for each treatment (control,
drought, control after recovery and recovery). An experimental unit was
represented by one plant in 15 L plastic pot.

\hypertarget{relative-water-content-membrane-damage-and-water-potential}{%
\subsubsection{Relative water content, membrane damage and water
potential}\label{relative-water-content-membrane-damage-and-water-potential}}

After harvesting, the total aboveground fresh matter (FM) of each plant
was determined. The relative water content (RWC) was assessed from leaf
samples immediately weighted and harvested (FM), followed by rehydration
in distilled water for 24 h in the dark (turgid mass. TM), and
transferred to an oven at \(60^o C\) until constant weight (dry weight,
DM). The relative water content was calculated according to Lima Neto et
al. (2015) by the equation: \[\frac{(FM - DM)} {(TM - DM)} \cdot 100\]

The membrane damage measured by electrolyte leakage was assessed as
described previously (Lima Neto et al., 2017). Leaf segments were placed
in test tubes containing 10 mL of deionized water and incubated for 24 h
in a shaking water bath. After that, the electric conductivity of the
medium was evaluated \((L_1)\). Then, the segments were boiled for 1 h,
cooled down to ambient temperature in an ice bath, and measured the
electric conductivity \((L_2)\). The membrane damage (MD) was calculated
as \(MD=\left(\frac{L_1} {L_2}\right) \cdot 100\) and expressed in \%.

The leaf predawn water potential \((\psi_w)\) was evaluated using a
pressure chamber (3000 Scholander PWSC, ICT, International, Armidale,
AUS) (Scholander, 1960). Due to the very short petiole, a fresh leaf as
wrapped in Parafilm and then excised a small amount of mesophyll
surrounding the mid-vein to fit the leaf into the chamber lid. This
adapted method has an artifact with reading consistently about 0.1 - 0.2
MPa less negative, see Edwards and Diaz (2006) .

\hypertarget{lipid-peroxidation-and-hydrogen-peroxide-content}{%
\subsubsection{Lipid peroxidation and hydrogen peroxide
content}\label{lipid-peroxidation-and-hydrogen-peroxide-content}}

The lipid peroxidation was assessed based on the formation of
thiobarbituric acid reactive substances (TBARS). TBARS concentration was
calculated using its absorption coefficient
\((155\; mM^{-1} \; cm^{-1})\) and expressed as
\(nmol\; MDA-TBA\; g^{-1}\; FM)\) (Cakmak and Horst, 1991). The
\(H_2O_2\) content was determined by the Amplex Red oxidation method.
Leaf segments were ground in a \(K^+\) phosphate buffer 100 mM, pH 7.5.
The crude extract was centrifuged at 12,000 rcf for 30 minutes at
\(4^o C\). Acoording to the manufacturer protocol, the supernatant was
supplemented with 10 mM Amplex-Red and 10 U of horseradish peroxidase.
The resorufin production was measured at 560 nm in a spectrophotometer
(Zhou et al., 1997).

\hypertarget{gas-exchange-and-chlorophyll-fluorescence}{%
\subsubsection{Gas exchange and chlorophyll
fluorescence}\label{gas-exchange-and-chlorophyll-fluorescence}}

Gas exchange parameters were measured using an infrared gas analyzer
(IRGA LI 6400 XT, LiCOR, Lincoln, NE, USA), with a leaf chamber
fluorometer attached (LI-6400-40, LICOR, Lincoln, NE, USA), on the
second fully expanded leaf. During the measurements, the conditions
inside the IRGA chamber were controlled to PPFD of 800
\(\mu mol\;m^{2}\;s^{-1}\), air \(CO_2\) partial pressure of 380
\(\mu mol \; mol^{-1}\), air vapor pressure deficit (VPD) or 1.0 \(\pm\)
0.5 kPa and air temperature of \(25^o C\). The amount of blue light was
set to 10\% of PPFD to maximize stomatal aperture (Ziotti et al., 2019).
Gas exchange was measured after at least 30 minutes of light exposure to
achieve steady-state photosynthesis after photosynthetic induction.

Chlorophyll \emph{a} fluorescence was measured using the saturation
pulse method with a portable chlorophyll fluorometer (JR-PAMIII WALZ,
Effeltrich, Germany). Leaves were dark-adapted for 30 min for assessing
Fo and Fm. Then, the leaves were exposed to actinic light
\((500\; \mu mol\; m^2\; s^{-1})\) for at least 30 min to reach a
photosynthetic steady state. The actinic light used was near the
saturation PPFD observed from previous light curves performed in our
laboratory. The intensity and duration of the saturation pulses were
\(8,000\; \mu mol\; m^2\; s^{-1}\) and 0.7 s, respectively. The
following parameters were assessed: the potential quantum yield of PSII
\([Fv/Fm = (Fm-Fo)/Fm]\); the effective quantum yield of PSII
\([YII = (Fm'- Fs)/Fm']\), the proportion of opened \((qP)\) and closed
\((1-qP)\) PSII states, the non-photochemical quenching
\([NPQ = (Fm – Fm')/Fm']\), the quantum yield of non-regulated
nonphotochemical energy loss in PSII \([Y(NO) = F/Fm]\) and the quantum
yield of regulated non-photochemical energy loss in PSII
\([Y(NPQ) = F/ Fm' - F/Fm]\). \(Fm\) and \(Fo\) are the maximum and
minimum fluorescence of dark-adapted leaves, respectively; \(Fm′\) and
\(Fs\) are the maximum and steady state fluorescence in the
light-adapted samples (Maxwell and Johnson, 2000).

\hypertarget{acid-titrations-and-pigment-contents}{%
\subsubsection{Acid titrations and pigment
contents}\label{acid-titrations-and-pigment-contents}}

Leaf samples were collected in triplicate. The main vein was avoided and
the leaf blade was frozen until assayed. The plant material was sampled
before dawn and after dusk at the end of the experiment. Individual
samples were ground in distilled water with a mortar and pestle. They
were then tritated with 0.01 N KOH using phenolphthalein solution as
indicator. Data are expressed as \(\mu eq\; acid\; g\; FM^{-1}\) (Rayder
and Ting, 1981).

Leaf segments were sampled and extracted in the dark at \(8^oC\) for 8h
in cold 80\% acetone solution. Then, the samples were centrifuged at
10,000 rcf for 5 min. The supernatant was read in a spectrophotometer at
different wavelengths as previously described (Lichtenthaler et al.,
1983). The total anthocyanin content was determined as previously
described by Silva et al. (2023). Leaf segments were extracted in
methanol and 1\% HCl at 4°C overnight. Subsequently, chloroform was
added to the homogenate, which was then centrifuged at 14,000 rcf for 5
min. The upper fraction was used for spectrophotometric reading at 530
nm and 657 nm. The total anthocyanin content was expressed as A530 -
A657 \(g^-1\) FW.

\hypertarget{protein-extraction-and-enzymatic-assays}{%
\subsubsection{Protein extraction and enzymatic
assays}\label{protein-extraction-and-enzymatic-assays}}

Leaf samples (1 g) were harvested and extracted in a cold mortar and
pestle with liquid \(N_2\) and 9 mL of the extraction medium consisting
of 100 mM Hepes-NaOH (pH 7.5), 2 mM \(MgCl_2\), 2 mM EDTA, and 10 mM
DTT. PVPP (0.5 g) was added before grinding each sample to aid mucilage
removal. The homogenate was squeezed through one layer of Miracloth. The
extract was centrifuged at 14,000 rcf at \(4^oC\) for 20 min before
being passed through a small Sephadex G-25 column previously
equilibrated with extraction buffer adjusted to pH 7.5. The extraction
method was adapted from a previous work of Rayder and Ting (1981).

\hypertarget{statistical-analysis}{%
\subsubsection{Statistical analysis}\label{statistical-analysis}}

\hypertarget{results}{%
\subsection{Results}\label{results}}

\hypertarget{discussion}{%
\subsection{Discussion}\label{discussion}}

\begin{itemize}
\tightlist
\item
  Briefly synthesize results
\end{itemize}

``This combination of CAM-like acid fluctuations with C3 gas exchange is
characteristic of CAM-cycling, albeit at low levels in these
pereskioids. Whether or not diel acid fluctuations or even CAM gas
exchange might be stimulated in these or the other species by drought
stress was not examined in this study.'' (Martin and Wallace, 2000).

Queiroz et al. (2015) propose the use of intermittent drought conditions
to encourage higher leaf area and photosynthetic efficiency in \emph{P.
aculeata}.

Therefore, further research is needed to understand the CAM
photosynthesis in \emph{P. aculeata} plants and its potential role in
helping the species to survive in changing environmental conditions
caused by climate change.

\hypertarget{conclusion}{%
\subsection{Conclusion}\label{conclusion}}

\hypertarget{ackonwledgments}{%
\subsection{Ackonwledgments}\label{ackonwledgments}}

\clearpage

\hypertarget{references}{%
\subsection{References}\label{references}}

\hypertarget{refs}{}
\begin{CSLReferences}{1}{0}
\leavevmode\vadjust pre{\hypertarget{ref-Cakmak1991}{}}%
Cakmak, I., Horst, W.J., 1991. Effect of aluminium on lipid
peroxidation, superoxide dismutase, catalase, and peroxidase activities
in root tips of soybean ({Glycine} max). Physiologia Plantarum 83,
463--468. \url{https://doi.org/10.1111/j.1399-3054.1991.tb00121.x}

\leavevmode\vadjust pre{\hypertarget{ref-chaves_understanding_2003}{}}%
Chaves, M.M., Maroco, J.P., Pereira, J.S., 2003. Understanding plant
responses to drought \textemdash{} from genes to the whole plant.
Functional Plant Biology 30, 239. \url{https://doi.org/10.1071/FP02076}

\leavevmode\vadjust pre{\hypertarget{ref-cousins_recent_2020}{}}%
Cousins, A.B., Mullendore, D.L., Sonawane, B.V., 2020. Recent
developments in mesophyll conductance in {C3}, {C4}, and crassulacean
acid metabolism plants. The Plant Journal 101, 816--830.
\url{https://doi.org/10.1111/tpj.14664}

\leavevmode\vadjust pre{\hypertarget{ref-edwards_ecological_2006-1}{}}%
Edwards, E.J., Diaz, M., 2006. Ecological physiology of {Pereskia}
guamacho, a cactus with leaves. Plant, Cell and Environment 29,
247--256. \url{https://doi.org/10.1111/j.1365-3040.2005.01417.x}

\leavevmode\vadjust pre{\hypertarget{ref-fenollosa_physiological_2019}{}}%
Fenollosa, E., Munné-Bosch, S., 2019. Physiological {Plasticity} of
{Plants Facing Climate Change}. Annual plant reviews online 2, 1--29.
\url{https://doi.org/10.1002/9781119312994.APR0686}

\leavevmode\vadjust pre{\hypertarget{ref-foyer_reactive_2018}{}}%
Foyer, C.H., 2018. Reactive oxygen species, oxidative signaling and the
regulation of photosynthesis. Environmental and Experimental Botany 154,
134--142. \url{https://doi.org/10.1016/j.envexpbot.2018.05.003}

\leavevmode\vadjust pre{\hypertarget{ref-gupta_physiology_2020}{}}%
Gupta, A., Rico-Medina, A., Caño-Delgado, A.I., 2020. The physiology of
plant responses to drought. Science 368, 266--269.
\url{https://doi.org/10.1126/science.aaz7614}

\leavevmode\vadjust pre{\hypertarget{ref-hoagland_water_1950}{}}%
Hoagland, D.R., Arnon, D.J., 1950. The water culture method for growing
plants without soil. Agric Exp Cir 347.

\leavevmode\vadjust pre{\hypertarget{ref-lawson_speedy_2019}{}}%
Lawson, T., Vialet-Chabrand, S., 2019. Speedy stomata, photosynthesis
and plant water use efficiency. New Phytologist 221, 93--98.
\url{https://doi.org/10.1111/nph.15330}

\leavevmode\vadjust pre{\hypertarget{ref-lichtenthaler_determination_1983}{}}%
Lichtenthaler, H., Wellburn, A., Lichtentharler, H.K., Welburn, A.R.,
1983. Determination of total carotenoids and chlorophylls a and b of
leaf extracts in different solvents. Biochemical Society Transactions
11, 591--592. \url{https://doi.org/10.1042/bst0110591}

\leavevmode\vadjust pre{\hypertarget{ref-lima_neto_jatropha_2015}{}}%
Lima Neto, M.C., Martins, M.O., Ferreira-Silva, S.L., Silveira, J.A.G.,
2015. Jatropha curcas and {Ricinus} communis display contrasting
photosynthetic mechanisms in response to enviromental conditions.
Scientia Agricola 72, 269--269.

\leavevmode\vadjust pre{\hypertarget{ref-lima_neto_regulation_2017}{}}%
Lima Neto, M.C., Silveira, J.A.G., Cerqueira, J.V.A., Cunha, J.R., 2017.
Regulation of the photosynthetic electron transport and specific
photoprotective mechanisms in {Ricinus} communis under drought and
recovery. Acta Physiologiae Plantarum 39, 183.
\url{https://doi.org/10.1007/s11738-017-2483-9}

\leavevmode\vadjust pre{\hypertarget{ref-luttge_ecophysiology_2004}{}}%
Luttge, U., 2004. Ecophysiology of {Crassulacean Acid Metabolism}
({CAM}). Annals of Botany 93, 629--652.
\url{https://doi.org/10.1093/aob/mch087}

\leavevmode\vadjust pre{\hypertarget{ref-martin_photosynthetic_2000}{}}%
Martin, C.E., Wallace, R.S., 2000. Photosynthetic {Pathway Variation} in
{Leafy Members} of {Two Subfamilies} of the {Cactaceae}. International
Journal of Plant Sciences 161, 639--650.
\url{https://doi.org/10.1086/314285}

\leavevmode\vadjust pre{\hypertarget{ref-maxwell_chlorophyll_2000}{}}%
Maxwell, K., Johnson, G.N., 2000.
\href{https://www.ncbi.nlm.nih.gov/pubmed/10938857}{Chlorophyll
fluorescence--a practical guide.} Journal of experimental botany 51,
659--68.

\leavevmode\vadjust pre{\hypertarget{ref-nobel_droughtinduced_1987}{}}%
Nobel, P.S., Hartsock, T.L., 1987. Drought-induced shifts in daily {CO2}
uptake patterns for leafy cacti. Physiologia Plantarum 70, 114--118.
\url{https://doi.org/10.1111/j.1399-3054.1987.tb06119.x}

\leavevmode\vadjust pre{\hypertarget{ref-queiroz_growing_2015}{}}%
Queiroz, C.R.A. dos A., de Andrade, R.R., de Morais, S.A.L., Pavani,
L.C., 2015. Growing {Pereskia} aculeata under intermittent irrigation
according to levels of matric potential reduction. Tropical agricultural
research 45, 1--8. \url{https://doi.org/10.1590/1983-40632015v4527210}

\leavevmode\vadjust pre{\hypertarget{ref-rayder_carbon_1981}{}}%
Rayder, L., Ting, I.P., 1981. Carbon {Metabolism} in {Two Species} of
{Pereskia} ({Cactaceae}). Plant Physiology 68, 139--142.
\url{https://doi.org/10.1104/pp.68.1.139}

\leavevmode\vadjust pre{\hypertarget{ref-rosenzweig_assessing_2014}{}}%
Rosenzweig, C., Elliott, J., Deryng, D., Ruane, A.C., Müller, C.,
Arneth, A., Boote, K.J., Folberth, C., Glotter, M., Khabarov, N.,
Neumann, K., Piontek, F., Pugh, T.A.M., Schmid, E., Stehfest, E., Yang,
H., Jones, J.W., 2014. Assessing agricultural risks of climate change in
the 21st century in a global gridded crop model intercomparison.
Proceedings of the National Academy of Sciences 111, 3268--3273.
\url{https://doi.org/10.1073/pnas.1222463110}

\leavevmode\vadjust pre{\hypertarget{ref-scholander_sap_1960}{}}%
Scholander, 1960. Sap {Pressure} in {Vascular Plants}. Science 148,
339--346.

\leavevmode\vadjust pre{\hypertarget{ref-silva_plasticity_2023}{}}%
Silva, B.P., Saballo, H.M., Lobo, A.K.M., Neto, M.C.L., 2023. The
plasticity of the photosynthetic apparatus and antioxidant responses are
critical for the dispersion of {Rhizophora} mangle along a salinity
gradient. Aquatic Botany 185, 103609.
\url{https://doi.org/10.1016/j.aquabot.2022.103609}

\leavevmode\vadjust pre{\hypertarget{ref-venter_interactive_2022}{}}%
Venter, N., Cowie, B.W., Paterson, I.D., Witkowski, E.T.F., Byrne, M.J.,
2022. The interactive effects of {CO2} and water on the growth and
physiology of the invasive alien vine {Pereskia} aculeata ({Cactaceae}):
{Implications} for its future invasion and management. Environmental and
Experimental Botany 194, 104737.
\url{https://doi.org/10.1016/j.envexpbot.2021.104737}

\leavevmode\vadjust pre{\hypertarget{ref-winter_ecophysiology_2019}{}}%
Winter, K., 2019. Ecophysiology of constitutive and facultative {CAM}
photosynthesis. Journal of experimental botany 70, 6495--6508.

\leavevmode\vadjust pre{\hypertarget{ref-winter_facultative_2014}{}}%
Winter, K., Holtum, J.A.M., 2014. Facultative crassulacean acid
metabolism ({CAM}) plants: {Powerful} tools for unravelling the
functional elements of {CAM} photosynthesis. Journal of Experimental
Botany 65, 3425--3441. \url{https://doi.org/10.1093/jxb/eru063}

\leavevmode\vadjust pre{\hypertarget{ref-zhou_stable_1997}{}}%
Zhou, M., Diwu, Z., Panchuk-Voloshina, N., Haugland, R.P., 1997. A
{Stable Nonfluorescent Derivative} of {Resorufin} for the {Fluorometric
Determination} of {Trace Hydrogen Peroxide}: {Applications} in
{Detecting} the {Activity} of {Phagocyte NADPH Oxidase} and {Other
Oxidases}. Analytical Biochemistry 253, 162--168.
\url{https://doi.org/10.1006/abio.1997.2391}

\leavevmode\vadjust pre{\hypertarget{ref-ziotti_photorespiration_2019}{}}%
Ziotti, A.B.S., Silva, B.P., Sershen, Lima Neto, M.C., 2019.
Photorespiration is crucial for salinity acclimation in castor bean.
Environmental and Experimental Botany 167, 103845.
\url{https://doi.org/10.1016/j.envexpbot.2019.103845}

\end{CSLReferences}

\newpage

\hypertarget{tables}{%
\subsection{Tables}\label{tables}}

\hypertarget{figures}{%
\subsection{Figures}\label{figures}}

\textbf{Figure 1.} The caption of figure 1

\newpage

\textbf{Figure 2.} The caption of figure 2



\end{document}
